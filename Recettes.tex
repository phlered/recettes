\documentclass[12pt,a5paper,openany]{book}
\usepackage[left=2cm,right=2cm,top=2cm,bottom=2cm]{geometry}
%\usepackage{makeidx}
%\makeindex
\usepackage[utf8]{inputenc} % encodage caracteres
\usepackage[english,francais]{babel}
\usepackage[T1]{fontenc}
\usepackage{aeguill}
%\usepackage{siunitx}
\renewcommand\familydefault{\sfdefault} \usepackage{hyperref} 




\title{\Huge{Mes Recettes}}
\author{Philippe-Henry Comte}

\begin{document}
\renewcommand{\contentsname}{MES RECETTES}

%\maketitle


\tableofcontents

\backmatter


\part*{RECETTES SAL\'EES} \addcontentsline{toc}{part}{RECETTES SAL\'EES} 



\chapter{Galettes de blé noir}
\begin{itemize}
\item 250 g de farine de blé noir
\item 1/2 cuillère à soupe de sel
\item 1 \oe uf
\item 0,5 l d'eau
\item 0,25 l de lait
\end{itemize}
Verser la farine dans une terrine en formant une fontaine\\
Y ajouter au centre le sel et l'\oe uf\\
Verser peu à peu eau et lait tout en délayant jusqu'à obtention d'une
pâte lisse et onctueuse.\\
Laisser reposer une heure.\\
Garnitures : \oe uf/jambon/fromage ; beurre-sucre ; crême et saumon
fumé






\chapter{Moelleux à la carotte}
Recette "diététique" ; pour un cake.
\begin{itemize}
\item 1 yaout nature
\item 2 pots de yaourt de farine
\item $\frac{1}{2}$ pot  d'huile
\item 3 \oe ufs
\item 180 g. de carottes râpées
\item 1 cuillère à soupe de noix de muscade \emph{ (ou si l'on préfère  une cuillère à café de quatre épices)}
\item  $\frac{1}{2}$ sachet de levure chimique
\end{itemize}
Mélanger l'ensemble des ingrédients. Faire cuire dans un moule à cake environ 30 minutes à 180 degrés








\chapter{Gougères au fromage}




Ingrédients (pour 8 personnes) : 
\begin{itemize}
\item  1 cuillère à café de sel
\item  150 g de gruyère râpé
\item  150 g de beurre
\item  300 g de farine
\item 50 cl d'eau
\item 8 oeufs
\end{itemize}

Faire une pâte à choux en faisant fondre dans une casserole le beurre, l'eau, le sel. Porter à ébullition.

Retirer du feu et ajouter la farine d'un seul coup en travaillant vivement à la spatule.

Remettre sur le feu et dessécher la pâte jusqu'à ce qu'elle se détache de la casserole.

Retirer du feu, laissez tiédir.

Incorporer les 8 oeufs, un à un, en mélangeant bien à la spatule.

Ajouter à cette pâte le gruyère râpé.

Disposer en petites boules sur une plaque beurrée.

Faire cuire à four chaud (210 degrés), jusqu'à ce que ça ait gonflé et que ce soit un peu doré (environ 25 min).




\chapter{Cake de Sophie aux olives}
\begin{itemize}
\item 3 \oe ufs
\item 150 g \ de farine
\item  1 sachet de levure
\item 10  cl d'huile de tournesol
\item  12,5 cl de lait
\item 100 g \ de gruyère râpé
\item  200 g \ de jambon de Paris
\item  75 g \ d'olives vertes
\item  sel, poivre
\end{itemize}
Cuire à 180 degrés \ pendant 45 minutes



\chapter{Pâte pour beignets fish and Chips}

\begin{itemize}
    \item 1 oeuf
    \item 125 g de farine
    \item 12,5 cl de bière
    \item sel
\end{itemize}





\chapter{Mini-Quiches sans pâte}
Recette "diététique"
\begin{itemize}
\item 100 ml de lait
\item 2 \oe ufs
\item 40 g \ de farine blanche
\item 40 g \ de champignons émincés
\item 50 g  \ de jambon
\item 40 g \  de gruyère râpé
\item 1 pincée de poivre
\end{itemize}
Faire cuire 15 minutes à 200 degrés puis 2 minutes au grill




\chapter{Pain de mie}

\begin{itemize}
    \item 900 g de farine
    \item 2 cuillère à soupe de sucre ou de miel
    \item 18 g de sel
    \item 350 ml de lait
    \item 250 ml d'eau
    \item 60 g beurre
\end{itemize}

Tous les ingrédients sauf beurre et levure 2mn vitesse 2 au pétrisseur. Attendre un peu, puis rajouter beurre et levure ; 10 mn de pétrissage vitesse 4.

Laisser monter jusqu'à ce que la pâte ait doublé de volume.
Puis dresser dans deux plats à cake.

Attendre de nouveau 1h.

Cuisson 25' à 180 degrés.



% \chapter{Rillettes de thon}
% Recette "diététique".  Pas vraiment testé...
% \begin{itemize}
% \item 140 g \ de thon égoutté
% \item 8 cornichons
% \item 100 g \ de fromage blanc
% \item Poivre
% \item 2 échalottes
% \item  10 tranches de pain complet
% \item  mélange d'herbes déshydratées
% \end{itemize}
% Couper finement les cornichons ainsi que les échalottes, le mélanger au reste. Déposer un peu de rillette de thon sur chaque tranche de pain. Ramener l'extrémité de la tranche et la fixer à l'aide de de 2 pics en bois




%%%%%%%%%%%%%%%%%%%%%%%%%%%%%%%%%%%%%%%%%%%%%%%%%%%%%
\part*{RECETTES SUCR\'EES} \addcontentsline{toc}{part}{RECETTES SUCR\'EES} 
%%%%%%%%%%%%%%%%%%%%%%%%%%%%%%%%%%%%%%%%%%%%%%

\chapter{Feuilles de brick aux amandes}

Pour 18 à 20 pièces (10 feuilles de brick)
\begin{itemize}
    \item 250 g poudre amande
    \item 170 g sucre
    \item 3 oeufs
\end{itemize}

Fourrer les demi-feuilles de brick, plier, dorer à la poele.
Avec un pinceau, recouvrir de miel chaud et laisser refroidir.



\chapter{Brownie au chocolat}
(Recette de Maud, du grand patissier)


% https://les-gourmandises-de-maud.fr/recettes/

\begin{itemize}
    \item $200 g$ de beurre doux
    \item $40 g$ d'huile neutre
    \item $460 g$ de chocolat noir
    \item $150 g$ de farine
    \item 6 oeufs
    \item $180 g$ de sucre
    \item 1 pincée de sel
    \item $100 g$ de noix de pécan ou noisettes ou pépites de chocolat
    
\end{itemize}

\begin{enumerate}
    \item  Fouettez les oeufs et le sucre, avec un fouet éléctrique environ $5 min$
    \item Faites fondre le chocolat noir et le beurre, en mélangeant les deux ensemble puis ajoutez l'huile.
    \item Ajoutez au mélange oeufs sucre qui doit avoir doublé de volume et mélangez à la maryse.
    \item Ajoutez la farine et le sel et mélangez.
    \item Beurrez le moule et versez la pâte.
    \item Ajoutez des noisettes et des noix de pécan sur le dessus.
    \item Enfournez $15 min$ dans un four préchauffé à $180^{\circ} C$.

\end{enumerate}






\chapter{Pâte brisée sucrée}

Par exemple pour faire une tarte aux pommes. Source = BacPro Besançon\\
\begin{itemize}
\item
250 g farine
\item 125 g beurre
\item 25  g sucre
\item 5  g sel
\item 50  g eau
\item 1 jaune \oe uf
\end{itemize}

Mélanger les ingrédients secs en sablant.\\
Puis intégrer eau et jaune \oe uf.\\
Laisser minimum une demi heure au frigo en filmant



\chapter{Tarte à la rhubarbe meringuée}

Pour une grande tarte :
\begin{itemize}
    \item Pâte brisée sucrée (à partir de 300 g de farine, 100 g beurre, 50 g eau, sel, sucre)
    \item 1,2 kg de rhubarbe
    \item 2 oeufs entiers + 2 jaunes pour le flan
    \item 1 sachet sucre vanillé
    \item 2 cuillères soupe MAIZENA (pas trop grosses)
    \item 30 cl crême fraîche
    
    \item sucre : 75 g pour la macération, 100 g dans le flan
    \item Meringque : 2 blancs (voire plus) et 40 g de sucre par blanc
\end{itemize}


La veille, laver les tiges, puis les
 éplucher 1 peau sur 2 (les 2 peaux gardées rendent la rhubarbe moins
 Couper les tiges en bâtonnets de $2 cm$ de long après avoir
fendu la tige en 2 ou 4 selon sa grosseur, saupoudrer de sucre et laisser
macérer 1 heure (moi je le fais souvent la veille).

Le lendemain :  battre les œufs entier, les  jaunes, le sachet de sucre vanillé,
les restant de sucre, la crème et la maizena. 

Etaler la pate brisée

Mettre ce flan sur la
rhubarbe, et enfourner pour 30 min. à 220

Battre les blancs en neige.
Quand la neige est formée, ajouter le sucre par petites quantités, en continuant
de battre. 
Lorsque la tarte est cuite, baisser
le four à $110^{\circ} C$ (thermostat 3-4).  Mettre les blancs en neige
sucrés dans la poche à douille, sortir la tarte du four et la garnir de cette
meringue en commençant par le centre en cercles concentriques ou toute autre
décoration à votre convenance. Laisser cuire au moins 1 h.

\chapter{Appareil à crème prise}

Par exemple pour recouvrir les pommes dans une tarte aux pommes

\begin{itemize}
\item 2 \oe ufs entiers
\item  1 jaune d'\oe uf
\item 50 g de sucre en poudre
\item 200 g de lait
\item 100 g de crême double
\item extrait de vanille
\end{itemize}
A mélanger tout simplement, puis verser dans la tarte.



\chapter{Financiers}

\begin{itemize}
    \item 75 de farine
    \item 225 g sucre poudre
    \item 110 g de beurre fondu
    \item 125 g de poudre d'amande
    \item 6 blancs d'oeufs
    \item 1 goutte de rhum
    \item 1 pincée de sel
\end{itemize}

Préchauffer le four à 190 degrés.

Mélanger tous les ingrédients sauf le beurre (inutile de monter les blancs en neige). Puis ajouter le beurre et mélanger. Verser dans des moules à financiers bien beurrés.

Cuire 15 minutes à 190 degrés.







\chapter{Tarte amandine poires chocolat}

\begin{itemize}
    \item Une grande boite de poire williams
    \item 150 g de chocolat noir
    \item 1 rouleau de pâte feuilletée
    \item 125 g de poudre d'amande
    \item 100 g de beurre
    \item 100 g de sucre
    \item 2 oeufs
\end{itemize}

Mélanger le beurre mou avec le sucre, la poudre d'amande. Battre les oeufs et
les rajouter au mélange.
Faire fondre le chocolat noir et l'étaler sur la pate feuilletée. Disposer les poires coupées en quartier, puis recouvrir du mélange.

Four moyen 15 minutes.

\chapter{Broyé du Poitou}

1 grand moule et un moyen

\begin{itemize}
\item
1,2 kg de farine
\item 600 g de beurre
\item 550 g de sucre
\item 1 blanc d'oeuf (garder le jaune pour le vernis)
\item 10 à 15 g de sel
\end{itemize}



\chapter{Fondant au chocolat}
\begin{itemize}
\item 5 oeufs
\item 5 cuillères à soupe de sucre
\item 3 cuillères à soupe de farine ou maïzena
\item 125 g \ de beurre
\item une tablette de chocolat noir
\end{itemize}
Faire fondre le beurre avec le chocolat. Ajouter au reste déjà mélangé. Cuisson environ 10 ou 15 minutes à 180 degrés



\chapter{Cookies au KitchenAid}

\begin{itemize}
    \item 700 g de farine
    \item 400 g de beurre
    \item 200 g de sucre roux
    \item 100 g de sucre blanc
    \item 2 oeufs
    \item 400 g de chocolat
    \item levure chimique
    \item sel
\end{itemize}

Mélanger dans le KitchenAid le beurre et le sucre jusqu’à ce que le mélange devienne crémeux. Une fois le mélange crémeux, il faudra y ajouter l’œuf, la farine, le sel et le bicarbonate de soude et mélanger le tout. La pâte doit être bien amalgamée avant de pouvoir y ajouter les pépites de chocolat.

Rouler un boudin de 5 cm de diamètre, et laisser reposer au frais.

Préchauffer à 180

Cuisson 1O minutes


\chapter{Rice crispies bar}

\textbf{Ingrédients}
\begin{itemize}
    \item 45 g de beurre
    \item 300 g de marshmallows
    \item 160 à 170 g de rice crispies
\end{itemize}
\bigskip

\textbf{Instructions}
\begin{itemize}
    \item Graisser un moule de 30 par 15 avec du beurre.
    \item Dans une grande casserole , fondre le beurre et les marshmallows
    \item Quand le mélange est fondu, ajouter les céréales hors du feu
    \item Bien mélanger puis transvaser dans le moule.
    \item Aplatir et laisser refroidir
\end{itemize}




\chapter{Clafoutis}


\begin{itemize}
    \item 1 kg de fruits (cerises, abricots \ldots)
    \item 3 oeufs entiers + 3 jaunes d'oeuf
    \item 150 g sucre
    \item 150 g farine
    \item 90 g beurre fondu
    \item vanille (éventuellement)
    \item 375 g lait
    \item pincée de sel
\end{itemize}
\bigskip

Disposer les fruits dans un grand moule beurré.
Mélanger progressivement les ingrédients puis verser sur les fruits
Cuire 45 minutes à 180 \degres





\chapter{Gaufres Lyonnaises}
\begin{itemize}
\item 500 g farine
\item une pincée de sel
\item 30 g de sucre
\item 10 g de levure
\item 125 g beurre
\item 3 à 4 \oe ufs
\item bière, lait, eau : total 3/4 de litres
\end{itemize}







\chapter{Visitandine}
\begin{itemize}
\item 6 blancs d'\oe ufs en neige
\item 125 g d'amandes
\item 150 g de farine
\item 300 g de sucre
\item 150 g beurre
\end{itemize}
Mélanger le beurre ramolli (pas fondu) + le sucre. \\
Ajouter les blancs en neige\\
Ajouter la poudre d'amande mélangée à la farine\\
Cuire 25/30 minutes à feu doux




\chapter{Pain d'épice}

(Ingrédients pour deux pains d'épices)

\begin{itemize}
    \item  450 g de miel
    \item  100 g de sucre brun
    \item  600g de farine de seigle
    \item  1 sachet levure chimique
    \item anis, muscade, cannelle, gingembre, 4 épices (1 cc de chaque)
    \item  4 oeufs 
    \item  3 carottes rapées
    \item  250 cl lait
    \item beurre : 150 g
    
\end{itemize}

Faites chauffer le  miel à la casserole  avec le beurre, le lait et les épices
  
Mélangez la farine avec la levure chimique.

Ajoutez le mélange chaud (en remuant idéalement avec une cuillère en bois).

Incorporez petit à petit 4 oeufs, amalgamer le tout

Préchauffez le four à 160°C (thermostat 5-6).

Versez la préparation dans deux moule à cake bien beurré et fariné.

Enfournez et laissez cuire pendant 1h

Démoulez le pain d'épices lorsqu'il a totalement refroidi. Attendez 24 heures au minimum avant de le déguster (le mieux est de le filmer)


\chapter{Cookies}

Pour deux plaques :

\begin{itemize}
    \item 3 oeufs
    \item 230 g sucre
    \item 250 g beurre
    \item 450 g farine
    \item 300 g pépite de chocolat
    \item 3 sachets de sucre vanillé
    \item 1 sachet de levure chimique
    \item 1 cc sel
\end{itemize}

Beurre ramolli + sucre + oeufs et sucre vanillé + farine, levure et sel + chocolat. Bien mélanger et former de petites boules.

Cuisson une dizaine de minutes à 180 degrés


\chapter{Galette des Rois}

\begin{itemize}
    \item 250 g de poudre d' amande
    \item 200 g de sucre
    \item 180 g de beurre mou
    \item un bouchon de rhum
    \item 3 oeufs
    \item une pincée de sel
    \item Deux pâtes feuilletées    
\end{itemize}

Mélanger le tout et déposer sur un cercle de pâte feuilletée sur un plaque en laissant environ
2 cm non recouvert au pourtour de la pâte.

Déposer la fève. Mettre de l'eau sur ce pourtour et
coller dessus le deuxième cercle.
Glaçage au jaune d'oeuf pur. 

Four à 180 °C et baisser à la fin de la cuisson (on peut mettre un peu de sucre
glace en fin de cuisson pour caraméliser)


\chapter{Tiramisu}

(pour 8 à 10 personnes, grand plat à gratin)

\begin{itemize}
    \item 1 kg de mascarpone
    \item 8 oeufs (ou 9 si petis)
    \item 100 g sucre poudre
    \item 4 sachets sucre vanillé
    \item Environ 40 biscuits cuilliere
    \item Quelques cuillères de cacao en poudre (100 \%)
    \item Café
\end{itemize}

Mélanger les jaunes et les sucres jusqu'à blanchiment. Incorporer la mascarpone, puis 2/3 des blancs
battus en neige avec une pincée de sel (la recette traditionnelle met tous les blancs, ce qui dégrade le goût). Etaler au fond du plat les biscuits
trempés dans du café, puis la moitié de l'appareil. Renouveler une deuxième
fois. Parsemer de cacao. Laisser reposer une demi-journée.


\chapter{Crême pâtissière façon mémé Arlette}
\begin{itemize}
\item 2 jaunes d'\oe uf + 1 \oe uf entier
\item 100 g sucre en poudre
\item 75 g farine
\item 1/2 l de lait
\item sucre vanillé
\item 50 g beurre
\end{itemize}
Battre les \oe ufs+sucre.\\
Incorporer farine\\
Bouillir lait+vanille.\\
Verser sur la préparation.\\
Réchauffer jusqu'à épaississement.\\
Rajouter le beurre hors du feu (ne plus remettre sur le feu).s







\chapter{Gateau au yaourt (recette de Géraldine)}
\begin{itemize}
\item 1 pot de yaourt
\item 3 \oe ufs
\item 2 pots de sucre
\item 1 paquet de sucre vanillé
\item 3/4 de pot d'huile
\item 3 pots de farine
\item 1 paquet de levure chimique
\end{itemize}
Cuire 30 à 35 min à th=225.\\
NB : meilleur si beurre fondu à la place de l'huile.




\chapter{Far breton}

\begin{itemize}
\item
220 g farine
\item 110 g sucre poudre (déja réduit ; inutile d'aller en dessous)
\item 1 sachet de sucre vanillé
\item 3/4 litre lait
\item  5 oeufs
\item 20 g beurre fondu
\item 400 g pruneau
\end{itemize}

Travailler comme une pate à crêpe. C'est très liquide mais à la cuisson devient normal !\\
Cuisson environ 1 heure à 180 degrés.



\chapter{Cheesecake}

\textbf{Pour la croûte :}

\begin{itemize}
\item 200 g de biscuits à thé émiettés
\item 80 g de beurre fondu
\end{itemize}

\bigskip

Préchauffer le four à 175 degrés. Mélanger les biscuits écrasés et le beurre fondu dans un moule 24 cm, tasser avec un verre, cuire 10 à 15 minutes.
Sortir la croute du four et augmenter la température à 225 degrés.

\bigskip

\textbf{Pour la crême :}
\begin{itemize}
\item 550 g de Philadelphia
\item 150 g de sucre en poudre
\item pincée de sel
\item 40 g de farine
\item 36 cl de crême fraîche épaisse
\item 5 oeufs moyens
\item 1 jaune d'oeuf
\item 1/2 cuillère à café d'extrait de vanille
\end{itemize}

\bigskip

En respectant l'ordre des ingrédients, mélanger, verser sur la croûte.
Enfourner 10 minutes. Sans ouvrir le four, baisser la température à 120 degrés et prolonger la cuisson de 1h ou 1H15. La crème doit rester légèrement tremblotante au milieu. Laisser refroidir au réfrigérateur pendant au moins 4 heures.







\chapter{Madeleines aux épices}
Recette "diététique" 
\begin{itemize}
\item 1 yaourt
\item 2 \oe ufs
\item $\frac{1}{2}$ sachet de levure chimique
\item  1 pot de yaourt de sucre
\item 2 pots de yaourt de farine
\item  $\frac{1}{2}$ pot d'huile
\item  2 cuillères à soupe de mélange pour pain d'épices	
\end{itemize}
Mettre dans des moules à madeleine et cuire 15 minutes à 200 degrés . Baisser le four à 120 degrés pendant 3 minutes puis les sortir du four.





\chapter{Tarte au citron (copine maman)}

POUR 6 PERSONNES (indication : multiplier par 1,5 les quantités de crême)

PATE\\
250 G DE FARINE\\
125 G DE BEURRE \\
4 CUILLEREES A SOUPE D EAU\\
ou utiliser la pate MN

CREME\\
2 CUILLEREES A SOUPE DE MAIZENA \\
150 G DE SUCRE\\
4 JAUNES D OEUFS\\
2 CITRONS NON TRAITES\\
125  G DE BEURRE \\
50 G D AMANDE EN POUDRE\\
 
MERINGUE\\
4 BLANCS D OEUFS\\
100G DE SUCRE GLACE\\
                                                    …............................................

FAIRE LA PATE ET LA LAISSER REPOSER 1 H
L ETALER 
METTRE DE L ALU DANS LE FOND ET RECOUVRIR D  UNE COUCHE DE HARICOT BLANC OU CHAINE SPECIALE OU CAILLOUX
FAIRE CUIRE 10 A 15 MN TH :7 (210 degrés)


TRAVAILLER LES 4 JAUNES D OEUFS AVEC 150 G DE SUCRE
 LE MELANGE DOIT ETRE BLANC ET MOUSSEUX
AJOUTER- LE ZESTE DES CITRONS
                  - LES AMANDES EN POUDRE
                  - LE BEURRE FONDU TIEDE
                  - LE JUS DES CITRONS
                  - DELAYER LES 2 C DE MAIZENA DANS UN PEU D EAU ET AJOUTER DELICATEMENT  A LA PREPARATION ATTENTION AUX GRUMEAUX



GARNIR LA TARTE AVEC LA CREME FAIRE CUIRE 20 MN  TH 6


BATTRE LES BLANCS D OEUFS AVEC UN PEU DE SEL
 AVANT QU'ILS SOIENT TROP FERMES  AJOUTER LES 100G DE SUCRE GLACE

RECOUVRIR LA TARTE DE CES BLANCS AVEC UNE POCHE A DOUILLE EN CERCLE CONCENTRIQUE
REMETTRE AU FOUR 10 MN ( LEGEREMENT DORE)




\chapter{Tarte à la citrouille}

\begin{itemize}
    \item 1 pâte brisée ou sablée
    \item 150 g
    de sucre
    \item 1 sachet
    de sucre vanillé
    
    % \item  1 pincée
        % de muscade 
    
    \item sel
    \item 800 g
    de citrouille
    \item 2
    oeufs entiers
    \item 2
    jaunes d'oeuf
    \item 20 cl
    de crème fraîche liquide
\end{itemize}


Faîtes cuire votre citrouille à la vapeur pendant 15 à 20 min.
Une fois cuite, en faire une purée.

Suite à cela, ajouter la crème, les oeufs, le sucre, la pincée de sel  (attention à ne pas cuire les oeufs)


Une fois le mélange bien homogène, il doit avoir l'air liquide.
Déroulez votre pâte dans un petit moule à tarte. Piquez la avec une fourchette.
Versez le mélange dans le plat à tarte.

Enfournez pendant 30 à 40 min suivant la puissance de votre four.

Déguster de préférence froid.




\chapter{Cake aux fruits confits}


\url{https://cuisinezavecdjouza.fr/cake-aux-fruits-confits-de-sophie-dudemaine/}

Ingrédients pour un moule à cake de $22 \mathrm{~cm}$
\begin{itemize}
    \item 3 œufs (150 gr environ)
    \item 125 gr de sucre glace ou de sucre en poudre
    \item 125 gr de beurre demi-sel
    \item 160 gr de farine tamisée
    \item 1/3 de sachet de levure chimique
    \item 100 gr de raisins de Corinthe
    \item 100 gr de cerises confites
    \item 50 gr de fruits confits
    \item 4 cuillerées à soupe de rhum (remplacé par le thé ou de la fleur d'oranger)
    \item 1 sachet de thé
    
\end{itemize}



\begin{enumerate}
    \item Préchauffer le four à $200^{\circ} \mathrm{C}$.
    \item Faire macérer ensuite les raisins secs dans un bol d'eau infusé d'un sachet de thé citron ou orange afin de gonfler.
    \item Crémer le beurre ramolli et le sucre au fouet jusqu'à ce que le mélange blanchisse.
    \item Ajouter les œufs un par un, puis incorporer la farine et la levure chimique. Travailler bien le mélange en ruban.
    \item Égoutter puis sécher les raisins gonflés dans du papier absorbant puis farinez les en retirant l'excédent.
    \item Incorporer alors tous les petits morceaux de fruits confits à la pâte en mélangeant bien la préparation.
    \item Verser dans un moule à cake beurré ou chemisé de papier sulfurisé.
    \item Avant d'enfourner, réduire la température du four à $180^{\circ} \mathrm{C}$ et cuire le cake pour 45 à 50 minutes environ.
    \item Si nécessaire, couvrir le cake d'un papier cuisson ou aluminium si sa surface est déjà bien dorée et que l'intérieur manque de cuisson.
    \item Pour vérifier sa cuisson, piquer avec la pointe d'un couteau ou cure dent le centre du cake.
    \item  A la sortie du four, badigeonner éventuellement d'un sirop de sucre parfumé ou d'un nappage neutre.
    
\end{enumerate}
\end{document}